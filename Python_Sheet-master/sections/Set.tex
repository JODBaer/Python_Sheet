\section*{Set}
%%%%%%%%%%%%%%%%%%%%%%%%%%%%%%%%%%%%%%%%%%%%%%%%%%
\hspace{1cm}
\rowcolors{1}{blue!10}{white}
\begin{tabular}{|l l l|}
	\hline set.add(elem) & set.intersection(*others) $\to$ set & set.remove(elem)
	\\ set.clear() & set.intersection\_update(*others) & set.symmetric\_difference(other) $\to$ set
	\\ set.copy() $\to$ set & set.isdisjoint(other) $\to$ bool & set.symmetric\_difference\_update(other)
	\\ set.difference(*others) $\to$ set & set.issubset(other) $\to$ bool & set.union(*others) $\to$ set
	\\ set.difference\_update(*others) & set.issuperset(other) $\to$ bool & set.update(*others)
	\\ set.discard(elem) & set.pop() $\to$ object &
	\\\hline
\end{tabular}
%%%%%%%%%%%%%%%%%%%%%%%%%%%%%%%%%%%%%%%%%%%%%%%%%%
\vspace{0.1cm}
\\
Ein set kann ebenfalls wie eine Liste behandelt werden, wobei die Werte nur einmalig vorkommen können und sie nicht veränderbar sind. Weitere Werte können hinzugefügt oder entfernt werden.
Sets sind ungeordnet, heisst, \\\textcolor{red}{die Reihenfolge der Elemente kann nicht vorhergesagt werden.}
\vspace{0.5cm}
\\
\vspace{0.1cm}
\textbf{Erzeugen}\\
\begin{minipage}[h]{10cm}
	\lstinputlisting{code/Set/Set_create.py}
\end{minipage}
\begin{minipage}[h]{8cm}
	\textcolor{red}{\textbf{Out:}}
	\\myset = {'I', 'cheese', 'love', 'swiss'}
	\\aset = {'six', 'two', 'one'}
\end{minipage}
%%%%%%%%%%%%%%%%%%%%%%%%%%%%%%%%%%%%%%%%%%%%%%%%%%
\\
\vspace{0.1cm}
\textbf{Basics}\\
\begin{minipage}[h]{10cm}
	\lstinputlisting{code/Set/Set_basic.py}
\end{minipage}
\begin{minipage}[h]{8cm}
	\textcolor{red}{\textbf{Out:}}
	\\myset = {12, 77, 'elem'}
	\\myset = set()
	\\aset = {'elem', 12, 77}
\end{minipage}
%%%%%%%%%%%%%%%%%%%%%%%%%%%%%%%%%%%%%%%%%%%%%%%%%%